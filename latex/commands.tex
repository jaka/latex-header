\newtheorem{izrek}{Izrek}[section]
\newtheorem{lema}[izrek]{Lema}
\newtheorem{trditev}[izrek]{Trditev}
\newtheorem{posledica}[izrek]{Posledica}
\newtheorem{aksiom}[izrek]{Aksiom}
\newtheorem{definicija}[izrek]{Definicija}
\theoremstyle{remark}
\newtheorem*{opomba}{Opomba}

\newcommand{\gv}[1]{\ensuremath{\text{\boldmath\(#1\)}}}	% for vectors of Greek letters

\DeclareMathOperator*{\cone}{Cone}	% stožec
\DeclareMathOperator*{\conv}{Conv}	% konveksna ovojnica
\DeclareMathOperator*{\diag}{diag}	% diagonalna matrika
\DeclareMathOperator{\coker}{coker}
\DeclareMathOperator*{\rank}{rank}	% rank preslikave
\DeclareMathOperator*{\supp}{supp}	% nosilec preslikave
\DeclareMathOperator*{\aff}{Aff}	% afina ovojnica
\DeclareMathOperator*{\dom}{dom}	% domena
\DeclareMathOperator*{\ext}{ext}	% zunanjost množice
\DeclareMathOperator*{\Int}{int}	% notranjost množice
\DeclareMathOperator*{\Lie}{Lie}	% Liejeva grupa
\DeclareMathOperator*{\im}{Im}		% imaginarni del
\DeclareMathOperator*{\re}{Re}		% realni del
\DeclareMathOperator*{\tr}{sl}		% sled
\DeclareMathOperator{\st}{deg}		% stopnja preslikave
\DeclareMathOperator*{\ord}{ord}	% veckratnost
\DeclareMathOperator{\lin}{Lin}		% linearna ovojnica
\DeclareMathOperator{\Aut}{Aut}		% množica avtomofrizmov
\DeclareMathOperator{\End}{End}		% množica endomorfizmov
\DeclareMathOperator{\ad}{ad}		% preslikava ad
\DeclareMathOperator{\Ad}{Ad}		% preslikava Ad

\newcommand{\cl}[1]{\overline{#1}} % zaprtje

\newcommand{\odp}[0]{\ ^{\mathrm{odp}}\!\!\subseteq}
\newcommand{\zap}[0]{\ ^{\mathrm{zap}}\!\!\subseteq}
\newcommand{\komp}[0]{\ ^{\mathrm{komp}}\!\!\subseteq}
\newcommand{\rn}[0]{\mathds{R}^n}		% evklidski prostor dimenzije n
\renewcommand{\rm}[0]{\mathds{R}^m}		% evklidski prostor dimenzije m
\renewcommand{\sp}[2]{\langle #1, #2 \rangle}	% skalarni produkt
\newcommand{\fp}[1]{\mathds{#1}\mathrm{P}}	% projektivni prostor
\newcommand{\cpn}[0]{\fp{C}^n}			% kompleksni evklidski prostor dimenzije n

\newcommand{\grad}[1]{\gv{\nabla}#1}				% gradient
\let\underdot=\d % rename builtin command \d{} to \underdot{}
\renewcommand{\d}[2]{\frac{\mathrm{d}#1}{\mathrm{d}#2}}		% odvod
\newcommand{\dd}[2]{\frac{d^2 #1}{d #2^2}}			% dvojni odvod
\newcommand{\pd}[2]{\frac{\partial #1}{\partial #2}}		% parcialni odvod
\newcommand{\pdd}[2]{\frac{\partial^2 #1}{\partial #2^2}}	% dvojni parcialni odvod
